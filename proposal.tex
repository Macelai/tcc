\documentclass{ufsctex/ufsctex}

\usepackage[table]{xcolor}
\usepackage{mdframed, enumitem, multirow, hhline, amssymb}

\newcolumntype{P}[1]{>{\centering\arraybackslash}m{#1}}

\newcommand{\thatcell}[3]{
  \multicolumn{#1}{#2}{\cellcolor{lightgray} \textbf{#3}}
}

\begin{document}

\instituicao[a]{Universidade Federal de Santa Catarina}
\departamento[o]{Departamento de Informática e Estatística}
\curso[o]{Programa de Graduação em Ciência da Computação}
\documento[a]{Monografia}
\titulo{Eleição eletrônica utilizando certificado digital auditada em blockchain}
\autor{Vinicius Macelai}
\grau{Bacharel em Ciência da Computação}
\local{Florianópolis}
\data{12}{novembro}{2018}
\orientador[Orientador]{Prof.\ Dr.\ Jean Everson Martina}

\textoResumo{
	Resumo bem daora
}
\palavrasChave{criptografia, eleição, blockchain}

\capa{}
\folhaderosto{}

\clearpage

\begin{mdframed}[backgroundcolor=lightgray, linewidth=0pt]
    \centering
    \textbf{FOLHA DE APROVAÇÃO DE PROPOSTA DE TCC}
\end{mdframed}

\vspace{-5mm}
\begin{table}[h]
  \begin{tabular}{|>{\bfseries}l|p{6.57cm}|}
    \hline
    Acadêmico(s)            & Vinicius Macelai                           \\ \hline
    Título do trabalho      & Eleição eletrônica utilizando
	  						certificado digital auditada em blockchain   \\ \hline
    Curso                   & Ciência da Computação/INE/UFS     C        \\ \hline
    Área de Concentração    & Segurança                                  \\ \hline
  \end{tabular}
\end{table}

\vspace{-2mm}
{\footnotesize\noindent\textbf{
  Instruções para preenchimento pelo \underline{ORIENTADOR DO TRABALHO}:} \\
  \begin{itemize}[leftmargin=3.6mm,label=-]
    \vspace{-4mm}
    \item Para cada critério avaliado, assinale um X na coluna SIM apenas
        se considerado aprovado. Caso contrário, indique as alterações
        necessárias na coluna Observação.
  \end{itemize}
}

\vspace{-3mm}
\begin{table}[hbpt]
  \begin{tabular}{|>{\tiny}m{4.3cm}*{4}{|>{\columncolor{lightgray}\tiny}c}|c|}
    \hline
    \rowcolor{lightgray} & \multicolumn{4}{c|}{\textbf{Aprovado}} & \\
    \hhline{|>{\arrayrulecolor{lightgray}}->{\arrayrulecolor{black}}|
        |---->{\arrayrulecolor{lightgray}}->{\arrayrulecolor{black}}|}
    \rowcolor{lightgray}
    \multicolumn{1}{|c|}{\multirow{-2}{*}{\normalsize\textbf{Critérios}}}
      & \textbf{Sim} & \textbf{Parcial} & \textbf{Não} & \textbf{Não se aplica}
      & \multirow{-2}{*}{\textbf{Observação}} \\ \hline
    1. O trabalho é adequado para um TCC no CCO/SIN
      (relevância / abrangência)?                         & & & & & \\ \hline
    2. O título do trabalho é adequado?                   & & & & & \\ \hline
    3. O tema de pesquisa está claramente descrito?       & & & & & \\ \hline
    4. O problema/hipóteses de pesquisa do trabalho
      está claramente identificado?                       & & & & & \\ \hline
    5. A relevância da pesquisa é justificada?            & & & & & \\ \hline
    6. Os objetivos descrevem completa e claramente
      o que se pretende alcançar neste trabalho?          & & & & & \\ \hline
    7. É definido o método a ser adotado no trabalho?
      O método condiz com os objetivos e é adequado
      para um TCC?                                        & & & & & \\ \hline
    8. Foi definido um cronograma coerente com o método
      definido (indicando todas as atividades) e com as
      datas das entregas (p.ex. Projeto I, II, Defesa)?   & & & & & \\ \hline
    9. Foram identificados custos relativos à execução
      deste trabalho (se houver)? Haverá financiamento
      para estes custos?                                  & & & & & \\ \hline
    10. Foram identificados todos os envolvidos neste
      trabalho?                                           & & & & & \\ \hline
    11. As formas de comunicação foram definidas
      (ex.: horários para orientação)?                    & & & & & \\ \hline
    12. Riscos potenciais que podem causar desvios do
      plano foram identificados?                          & & & & & \\ \hline
    13. Caso o TCC envolva a produção de um software ou
      outro tipo de produto e seja desenvolvido também
      como uma atividade realizada numa empresa ou
      laboratório, consta da proposta uma declaração
      (Anexo 3) de ciência e concordância com a entrega
      do código fonte e/ou documentação produzidos?       & & & & & \\ \hline
  \end{tabular}

  \vspace{2mm}
  {\footnotesize
  \begin{tabular}{|>{\bfseries}p{3cm}|l|l|l|}
    \hline Avaliação & \multicolumn{2}{l}{\bf $\square$ Aprovado}
      & \textbf{$\square$ Não Aprovado} \\
    \hline Professor Responsável & Jean Everson Martina & 12/11/2018 & \\
    \hline
  \end{tabular}}
\end{table}

\paginaresumo

\sumario

\chapter{Introdução}


\section{Objetivos}

\noindent \emph{Objetivo geral.} Apresentar um estudo detalhado sobre esquemas
\\

\noindent \emph{Objetivos específicos.} Descrever os esquemas de assinatur
\\

\noindent \emph{Escopo do trabalho.} Não se aplica ao conteúdo deste trabalho
\\

\noindent \emph{Critérios de aceitação.} Estudo e implementação de pelo menos
\\

\noindent \emph{Entregas do projeto.} Relatórios referentes às disciplinas 
\\

\noindent \emph{Restrições e premissas.} Espera-se reunir com os orientadores,

\section{Procedimentos metodológicos}

O trabalho será desenvolvido utilizando a infraestrutura e recursos do
Laboratório de Segurança em Computação (LabSEC/UFSC), onde será estudada
bibliografia referente aos temas abordados nesta pesquisa buscando encontrar
as vantagens e desvantagens entre cada um dos esquemas de assinatura digital
escolhidos, bem como observar seu desempenho e tamanho de elementos como
par de chaves e assinatura, ao utilizar funções de resumo criptográficas
distintas em implementações produzidas ou fornecidas.

\chapter{Cronograma}

\begin{figure}[htbp]
  \begin{tabular}{|p{4.04cm}|*{6}{c|}}
    \hline \rowcolor{lightgray}
      & \multicolumn{6}{c|}{\textbf{Meses -- 2018}} \\
    \hhline{|>{\arrayrulecolor{lightgray}}->{\arrayrulecolor{black}}|
      |------>{\arrayrulecolor{lightgray}}>{\arrayrulecolor{black}}|}
    \rowcolor{lightgray}
      \multicolumn{1}{|c|}{\multirow{-2}{*}{\textbf{Etapas}}}
      & \textbf{jan.} & \textbf{fev.} & \textbf{mar.}
      & \textbf{abr.} & \textbf{mai.} & \textbf{jun.} \\
    \hline Fundamentação teórica & \cellcolor{lightgray} & & & & & \\
    \hline Revisão do estado da arte & \cellcolor{lightgray}
      & \cellcolor{lightgray} & & & & \\
      \hline Desenvolvimento do TCC & & \cellcolor{lightgray}
      & \cellcolor{lightgray} & \cellcolor{lightgray} & & \\
    \hline Implementação & & & & \cellcolor{lightgray}
      & \cellcolor{lightgray} & \cellcolor{lightgray} \\
    \hline Relatório de TCC I & & & & & \cellcolor{lightgray} & \\
    \hline
    \hline \rowcolor{lightgray}
      & \multicolumn{6}{c|}{\textbf{Meses -- 2018}} \\
    \hhline{|>{\arrayrulecolor{lightgray}}->{\arrayrulecolor{black}}|
      |------>{\arrayrulecolor{lightgray}}>{\arrayrulecolor{black}}|}
    \rowcolor{lightgray}
      \multicolumn{1}{|c|}{\multirow{-2}{*}{\textbf{Etapas}}}
      & \textbf{jul.} & \textbf{ago.} & \textbf{set.}
      & \textbf{out.} & \textbf{nov.} & \textbf{dez.} \\
    \hline Ajustes na implementação & \cellcolor{lightgray} & & & & & \\
    \hline Redação da monografia & \cellcolor{lightgray}
      & \cellcolor{lightgray} & \cellcolor{lightgray} & & & \\
    \hline Ajustes na monografia & & & \cellcolor{lightgray}
      & \cellcolor{lightgray} & & \\
    \hline Relatório de TCC II & & & & & \cellcolor{lightgray} & \\
    \hline Defesa pública & & & & & & \cellcolor{lightgray} \\
    \hline Ajustes finais do TCC & & & & & & \cellcolor{lightgray} \\
    \hline
  \end{tabular}
\end{figure}

\chapter{Custos}

\begin{figure}[htbp]
  \begin{tabular}{|p{1.69cm}|*{3}{l|}}
    \hline \thatcell{1}{|c|}{Item} & \thatcell{1}{c|}{Quantidade}
      & \thatcell{1}{c|}{Valor unitário (R\$)}
      & \thatcell{1}{c|}{Valor Total (R\$)}                         \\
    \hline \thatcell{4}{|l|}{Material permanente}                   \\
    \hline Computador   & 1     & R\$ 2.500,00  & R\$ 2.500,00      \\
    \hline Internet     & 1     & R\$ 1.000,00  & R\$ 1.000,00      \\
    \hline Artigos      & 4     & R\$ 50,00     & R\$ 200,00        \\
    \hline \thatcell{4}{|l|}{Material de consumo}                   \\
    \hline Mídia óptica & 4     & R\$ 2,00      & R\$ 8,00          \\
    \hline \thatcell{4}{|l|}{Outros recursos e serviços}            \\
    \hline Impressões   & 200   & R\$ 0,20      & R\$ 40,00         \\
    \hline
  \end{tabular}
\end{figure}

\chapter{Recursos Humanos}

\begin{figure}[htbp]
  \begin{tabular}{|*{2}{p{4.96cm}|}}
    \hline \rowcolor{lightgray}
    \thatcell{1}{|c|}{Nome}       & \thatcell{1}{c|}{Função}    \\
    \hline Vinicius Macelai       & Autor                       \\
    \hline Jean E. Martina        & Orientador                  \\
    \hline Renato Cislaghi        & Coordenador de projetos     \\
    \hline A definir              & Membro(s) da banca          \\
    \hline
  \end{tabular}
\end{figure}

\chapter{Comunicação}

\begin{figure}[htbp]
  \footnotesize
  \begin{tabular}{|P{1.9cm}|P{0.7cm}|P{2.5cm}|P{1.5cm}|P{2cm}|}
    \hline \rowcolor{lightgray}
    \textbf{O que precisa ser comunicado} & \textbf{Por quem}
      & \textbf{Para quem} & \textbf{Melhor forma de comunicação}
      & \textbf{Quando e com que frequência} \\
    \hline Entrega do projeto do TCC & Autor
      & Orientador, coorientador, coordenador de projetos & Sistema de TCC
      & Uma vez, até dia 12/11/2018 \\
    \hline Entrega de relatório de TCC I & Autor
      & Orientador, coorientador, coordenador de projetos,
      membro(s) da banca & Sistema de TCC
      & Uma vez, ao final do semestre 2019/1 \\
    \hline Entrega de relatório de TCC II & Autor
      & Orientador, coorientador, coordenador de projetos,
      membros(s) da banca & Sistema de TCC
      & Uma vez, aproximadamente na metade do semestre 2019/2 \\
    \hline Defesa do TCC & Autor
      & Orientador, coorientador, coordenador de projetos,
      membro(s) da banca & Pessoalmente
      & Uma vez, aproximadamente na metade do semestre 2019/2 \\
    \hline Entrega final da monografia & Autor
      & Orientador, coorientador, coordenador de projetos,
      membro(s) da banca & Sistema de TCC
      & Uma vez, após a defesa, antes do término de 2019/2 \\
    \hline Reuniões de acompanhamento do desenvolvimento & Autor
      & Orientador, coorientador & Pessoalmente, webconferência
      & Quinzenalmente \\
    \hline Monitorar o projeto & Autor
      & Orientador, coorientador & E-mail & Eventualmente \\
    \hline Convite de membro(s) da banca & Autor & A definir
      & Sistema de TCC & Uma vez, em meados do semestre 2019/1 \\
    \hline
  \end{tabular}
\end{figure}

\chapter{Riscos}

\begin{figure}[htbp]
  \footnotesize
    \begin{tabular}{|P{1.3cm}|*{3}{P{0.8cm}|}*{2}{P{2.24cm}|}}
    \hline \rowcolor{lightgray}
    \textbf{Risco} & \textbf{Proba\-bilidade} & \textbf{Impacto}
      & \textbf{Priori\-dade} & \textbf{Estratégia de resposta}
      & \textbf{Ações de prevenção} \\
	\hline Alteração de tema & Baixa & Alto & Alta
	  & Alterar o escopo do tema, ou modificar completamente o tema.
	  & Ter interação constante com o orientador. \\
    \hline Paralisação dos servidores da UFSC & Baixa & Alto & Média
	  & Desenvolver o trabalho e pesquisa utilizando recursos próprios
      & Não se aplica \\
    \hline Problemas de saúde & Baixa & Alto & Alta
	  & Procurar ajuda especializada para tratar corretamente
	  & Evitar ambientes de exposição em caso de doenças, fazer exames
		para checar condição atual. \\
    \hline Perda de dados  & Muito baixa & Alto & Média
      & Recuperar as cópias armazenadas na nuvem
	  & Realizar backup de forma periódica de todo desenvolvimento \\
    \hline Falha nos equipamento(s) & Muito baixa & Alto & Média
      & Comprar novo(s) equipamento(s)
      & Evitar utilização do(s) equipamento(s) em más condições de tempo ou
        por períodos muito prolongadosa \\
    \hline
  \end{tabular}
\end{figure}

\bibliographystyle{abnt-alf}
\bibliography{ref}

\end{document}
