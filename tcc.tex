\documentclass{ufsctex/ufsctex}

\usepackage[table]{xcolor}
\usepackage{mdframed, enumitem, multirow, hhline, amssymb}

\newcolumntype{P}[1]{>{\centering\arraybackslash}m{#1}}

\newcommand{\thatcell}[3]{
	\multicolumn{#1}{#2}{\cellcolor{lightgray} \textbf{#3}}
}

\begin{document}

\instituicao[a]{Universidade Federal de Santa Catarina}
\departamento[o]{Departamento de Informática e Estatística}
\curso[o]{Programa de Graduação em Ciência da Computação}
\documento[a]{{Trabalho de Conclusão de Curso}}
\titulo{Eleição eletrônica utlizando blockchain e certificado digital}
\autor{Vinicius Macelai}
\grau{Bacharel em Ciência da Computação}
\local{Florianópolis}
\data{01}{julho}{2019}
\orientador[Orientador]{Prof.\ Dr.\ Jean Everson Martina}

\textoResumo{
	As abordagens utilizadas  nos sistemas de eleição da maioria dos países continuam
	a ser realizadas de  forma manual, com cédulas na forma de papel. Tal modelo traz
	problemas enormes de logística e um alto custo para funcionamento devido aos 
	requisitos de uma eleição segura, que deve fornecer privacidade, transparência,
	verificabilidade e confiabilidade. Já as  abordagens eletrônicas  via internet,
	permanacem com desconfiança	sobre manter estas propiedades.
	Uma possível solução para melhorar uma abordagem eletrônica seria utilizar uma blockchain
	para melhorar sua auditabilidade, que é o ponto mais questionado nesse esquema. A blockchain
	possui propiedades intrínsecas, como a imutabilidade dos dados, e com esquemas utilizando
	contratos inteligentes na blockchain é possível realizar a verificação dos votos de forma 
	descentralizada e  aberta ao público. Assim, cria-se um sistema que mantém as propiedades
	citadas anteriormente, com um baixo custo e menor necessidade de confiar em uma entidade central.
}
\palavrasChave{criptografia, eleições, democracia, blockchain, privacidade}

\capa{}
\folhaderosto{}

\clearpage

\paginaresumo

\sumario

\chapter{Introdução}
Ainda que vivemos no momento onde tudo é digital e fazemos as mais diversas tarefas de maneira
eletrônica e online, quando o assunto é votação no meio eletrônico, existem as mais diversas e
controversas opiniões a respeito.

No Brasil eleições eletrônicas têm sido utlizadas por mais de 20 anos e testes recentes mostram
que, mesmo com o desenvolvimento durante todo esse período, o atual sistema não consegue
se mostrar realmente seguro \cite{aranha}. O maior problema com a solução proposta pelo
Governo Brasileiro é a falta de auditabilidade, em que só é possível se voluntariar para testar
o sistema em um ambiente controlado. Ainda durante o processo eleitoral, é necessário confiar
cegamente no sistema, não há instrumento nenhum que permita verificar se o voto foi
realmente computado.

Os sistemas de votação eletrônicos atuais se baseiam em esquemas que utilizam de criptografia
homomórfica, que permite que dados cifrados possam ser processados sem serem decifrados, assim
garantidos propriedades importantes para o sistema.\cite{springer}. Entretanto, esses esquemas
são utilizados de forma centralizada, rodando apenas em um servidor central, sem a possiblidade
de tais informações serem acessadas para o público em geral de maneira transparente.

Uma maneira de se realizar a autencicação no sistema seria utilizando certificado
digital, que são arquivos digitais, que permite que uma pessoa seja identificada
virtualmente, com garantia de autenticidade \cite{pki}. Sendo no Brasil, o modelo adotado para
gerenciar o sistema, a Infraestrutura de Chaves Públicas Brasileira (ICPBrasil). Já na área
da educação, há a Infraestrutura de Chaves Públicas para Ensino e Pesquisa (ICPEdu), que
pode ser utilizada em votações no âmbito acadêmico.

Para realizar a auditoria das votações, é possível utilizar da tecnologia
blockchain, que são bases de registro de dados distrubuídos e compartilhados,
desta forma criando um consenso e confiança. \cite{nakamoto2012bitcoin} Com essas
propriedades intrínsecas, como a imutabilidade dos dados, é um bom sistema para 
manter o registro dos votos, além da possiblidade de rodar contratos inteligentes que
podem processar os votos de maneira decentralizadas junto com a criptografia homomórfica
para garantir o anonimato.

Neste trabalho foi optado por enfatizar o estudo e a utilização da blockchain e protocolo
Ethereum, a qual fornece contratos inteligentes de alto nível. É apresentado um esquema
que utliza esses contratos para garantir as propriedades já ditas, além do estudo de seu
impacto financeiro.

Este trabalho visa implementar uma solução que integre todas estas partes, um sistema
de eleição eletrônico autenticado com certificado digital que permite ter informações
sobre o votante de forma confiável. Além disso, realizar a verificação da eleição em 
uma blockchain de forma descentralizada e pública.

\section{Motivação}

\section{Pergunta de pesquisa}

\section{Hipóteses}

\section{Objetivo geral}

Estudar e criar uma implementação de um sistema
online de eleição, com seu foco principal na parte de realizar auditoria e
verificação dos votos em uma blockchain, sendo possível utilizar de um sistema
já desenvolvido que suporte autenticação com certificado digital. Além disso,
analisar as implicações que esse sistema teria no funcionamento e custos de
uma eleição. \\

\section{Objetivo específico}

\begin{itemize}
	\item Analisar o estado da arte: estudar as principais soluções
	já propostas na literatura, com o objetivo de identificar problemas
	e oportunidades para melhorar o trabalho.
	\item Implementar autenticação com certificado digital: Utilizar
	de um sistema já existente de votação e implementar a possiblidade
	de autenticar com certificado digital.
	\item Implementar possibilidade de verificação na blockchain:
	Implementar um sistema para tornar o sistema mais auditável e 
	verificável para o público em geral.
	\item Comparar e analisar as consequências do esquema.
\end{itemize}

\section{Metodologia}

O trabalho será desenvolvido utilizando a infraestrutura e recursos do
Laborarório de Segurança em Computação (LabSEC/UFSC), onde será estudada
a bibliografia referente aos assuntos abordados nesta pesquisa, visando
encontrar uma abordagem para um sistema de eleição eletrônica utilizando
certificação digital juntamente com blockchain. Frisando suas vantagens e
desvantanges juntamente com seus custos.

\section{Resultados esperados}

\section{Descrição do trabalho}



\bibliographystyle{abnt-alf}
\bibliography{ref}

\end{document}
